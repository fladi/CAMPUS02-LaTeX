% vim: set tw=160:

% 2010 Peter A. Menzel

\noindent
\begin{minipage}{\textwidth}
%%% minipage um Kapitel Kurzfassung UND Kapitel Abstract
%%% auf eine Seite zu binden
\chapter*{Kurzfassung}
Die Zunahme an Websites, auf denen es möglich ist, auch als Benutzer Inhalte zu veröffentlichen, macht es immer wichtiger, diese Inhalte auch eindeutig, über
die Grenzen einer einzelnen \glstext*{website} hinweg, einem Autor eindeutig zuordnen zu können. Um diese Zuordnung zuverlässig herstellen zu können, bieten
sich die drei kryptographischen Eigenschaften Integrität, Authentizität und Verbindlichkeit in der Form von digitalen Signaturen an. Dazu betrachtet diese
Arbeit Anforderungen, die an digitale Signaturen in einer \glstext*{website} gestellt werden. Es werden Ansätze zur Verbindung von \gls{website} und digitaler
Signatur erläutert und die Verfahren \gls{openpgp} und \gls{x509} auf ihre Tauglichkeit für diesen Einsatzzweck gegenübergestellt. Als Resultat werden aus
diesen Informationen mehrere Anforderungen für ein \glstext{firefox-addon}, welches \glstext*{xml-dsig} implementiert, abgeleitet.

\chapter*{Abstract}
With the growth of websites that allow their users to publish their own content, it is becoming more and more important that content can be attributed to an
author even beyond the boundaries of a single website. In order to establish such a reliable mapping, it would be desirable to apply the three cryptographic
properties of integrity, authenticity and non-repudiability in the form of digital signatures to any content on a website. This thesis investigates the
requirements digital signatures, as defined by the \glstext*{xml-dsig} standard, impose on a website. First, approaches for embedding digital signatures within a
website are identified. Then, two digital signature systems, \glstext*{openpgp} and \glstext*{x509}, are evaluated to determine their suitability for a website
scenario. Finally, the results of this evaluation are used to derive the requirements for the future development of a Firefox add-on that would incorporate
\glstext*{xml-dsig} to validate digital signatures in websites.

\end{minipage}
