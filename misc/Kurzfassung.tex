% vim: set tw=160:

% 2010 Peter A. Menzel

\noindent
\begin{minipage}{\textwidth}
%%% minipage um Kapitel Kurzfassung UND Kapitel Abstract
%%% auf eine Seite zu binden
\chapter*{Kurzfassung}
Diese Arbeit wurde als Theoriefindung zum Thema ``Digitale Signaturen in \glstext*{xhtml}'' geschrieben und behandelt die Anforderungen, die an eine Verbindung
der beiden Technologien \glstext*{xhtml} und \glstext*{xml-dsig} gestellt werden. Es wird die Funktionsweise von \glstext*{xml-dsig} beschrieben, und die damit
verwendbaren, bereits im Internet verbreiteten Verfahren zu Erstellung von digitalen Signaturen sowie zur Verteilung von Zertifikaten beziehungsweise
öffentlichen, asymmetrischen Schlüsseln, \glstext*{openpgp} und \glstext*{x509}, erläutert. Für die Verbindung zwischen \glstext*{xml-dsig} und
\glstext*{xhtml} werden mehrere Ansätze vorgestellt, die das Vorhandensein einer Signatur auf der Ebene von \glstext*{http}/\glstext*{https} oder \gls{xhtml}
anzeigen. Diese Informationen werden letztlich zu Anforderungen an ein, noch zu entwickelndes, AddOn für einen \glstext*{webbrowser}, welcher \gls{openpgp} und
auch \gls{x509} für seine kryptografische Funktionalität benutzt, zusammengefasst.

\chapter*{Abstract}
\blindtext

\end{minipage}
