% vim: set spell spelllang=de_at,en_us tw=160

\chapter{Einleitung}
\index{Einleitung}%
\label{chp:Einleitung}

\section{Ausgangssituation}
\index{Ausgangssituation}%
\label{sec:Einleitung:ausgangssituation}
Kryptographische Eigenschaften wie Vertraulichkeit, Authentizität und Integrität von digitalen Inhalten gewinnen in einem wachsenden globalen Netzwerk immer
stärker an Bedeutung. Dies lässt sich anhand der Verbreitung von \gls{ssl}-fähigen Webservern und den darauf eingesetzten Zertifikaten bestimmen. 
\gls{ssl} dient jedoch nur der Vertraulichkeit sowie der Integrität aller übertragenen Daten und der Authentizität zwischen den Peers einer unidirektionalen
Komunikation, wie sie z.B. bei HTTP zum Einsatz kommt. Aktive (umleiten der Verbindung) und passive Angriffe (Lesen der übertragenen Daten) wierden dadurch
verhindert, eine Prüfung der Integrität und Authentizität einzelner, festlegbarer Bereiche des Inhalts ist damit jedoch nicht möglich. \\

In anderen Bereihen der Kommunikation wie Email und \gls{im} gehören Verfahren zum Nachweis der Authentizität des Absenders und der Integrität der Nachricht
bereits zum Leitsungsumfang vieler Anwendungsprogramme. Die Verfahren welche dabei meist zum Einsatz kommen sind \gls{pgp} und \gls{smime}. Sie bieten die
Möglichkeit, digitale Signaturen auf Basis von asymmetrischen Verschlüsselungsverfahren zu erstellen. Dabei wird von Verfasser der Nachricht diese mit seiner
digitalen Signatur versehen. Der Empfänger ist nun in der Lage, die in der Nachricht signierten Teile zu überprüfen. \\

Mit dem Aufkommen des sog. Web 2.0 entstand auf vielen Webseiten für die Besucher die Möglichkeit, eigene Inhalte einzubinden oder bestehende Inhalte weiter zu
verwenden. Der Betreiber einer Seite wurde dadurch vom alleinigen Bereitsteller zu einem Verwalter von Informationen. In Bezug auf Authentizität, Integrität und
Verbindlichkeit des Inhalts waren Besucher somit meist auf optische Merkmale zu Separation von redaktionellem und Benutzer-generiertem Inhalt. \\

Digitale Signaturen könnten genau dieses Bedürfnis nach kryptografisch sicherer Auszeichnung von Inhalten in einem \gls{xhtml}-Dokument erfüllen.

\section{Aufgabenstellung}
\index{Aufgabenstellung}%
\label{sec:Einleitung:aufgabenstellung}
Die primäre Aufgabenstellung dieser Arbeit besteht darin, zu bestimmen, welche Anforderungen an eine Integration von digitalen Signaturen in
\gls{xhtml}-Dokumente gestellt wird. Folgende Aspekte sollen deshalb in dieser Arbeit erläutert werden:
\begin{itemize}
    \item Wie können digitale Signaturen in \gls{xhtml} integriert werden, ohne dass der entsprechende W3C-Standard\cite{xhtml:w3c} erweitert oder angepasst
    werden?
    \item Welche Möglichkeiten von XML eignen sich zur wahlfreien Selektion von Inhalten innerhalb eines \gls{xhtml}-Dokuments?
    \item Welche Implementierung eignet sich für die Integration in einen Browser?
    \item Wie kann die Verteilung der öffentlichen Schlüssel sowie die Referenzierung auf diese gelöst werden? Welche der bestehenden Technologien eigent sich
    dafür am besten?
\end{itemize}

\section{Zielsetzung der Arbeit}
\index{Ziele}
\label{sec:Einleitung:ziele}
Diese Arbeit wird die Eignung der vorhandenen Signaturverfahren für \gls{xml}, Möglichkeiten zur Schlüsselverteilung und Präsentation der Signaturen im Browser behandeln.
Nicht behandelt wird die Erstellung eines neuen Signaturverfahrens sowie eines neuen Signaturmodells für \gls{xml} oder die Erweiterung des bestehenden \gls{xhtml}-Standards. Auch ist eine Implementierung eines Browser-Plugins nicht im Rahmen dieser Arbeit vorgesehen.

\section{Vorgehen und Methodik}
\index{Methodik}
\label{sec:Einleitung:methodik}
\Blindtext

\section{Aufbau der Arbeit}
\index{Aufbau der Arbeit}
\label{sec:Einleitung:aufbau}
\Blindtext

