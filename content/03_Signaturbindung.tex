% vim: set tw=160:

\chapter{Signaturen mit XML-DSig}
\index{Signaturen mit XML-DSig}
\label{chap:SignaturenXMLDSig}
\gls{xml-dsig} definiert ein \gls{xml}-Format das Informationen über die zu signierenden Inhalte, die Signaturen selbst und die zur Prüfung der Signaturen
notwendigen Schlüsselinformationen aufnimmt. Um \gls{xml-dsig} von anderen \gls{xml}-Formaten abzugrenzen wurde ihm der \gls{xml}-Namensraum
\url{http://www.w3.org/2000/09/xmldsig#} zugewiesen. 


\section{Struktur einer Signatur}
Eine Signatur mit \gls{xml-dsig} besteht aus mehreren Elementen die sich alle innerhalb von \xmlelem{Signature} befinden. \fref{lst:xml-dsig-structure} zeigt
die grundlegende Struktur einer Signatur mit \gls{xml-dsig}.

\lstinputlisting[language=XML,caption={Struktur von \texorpdfstring{\protect\Glsentryname{xml-dsig}}{XML-DSig}},label=lst:xml-dsig-structure,emph={Signature}]{source/xml-dsig-structure.xml}

Im folgenden werden die einzelenen Elemente dieser Struktur in ihrer Funktion näher beschrieben. 

\section{Schlüsselinformation}
Das optionale Element \xmlelem{KeyInfo} enthält Informationen zur Schlüssel, der bei der Erstellung der Signatur verwendet wurde. Diese Informationen sollen es
ermöglichen, dass der Empfänger die notwendigen Informationen zur Validierung der Signatur beziehen kann. \gls{xml-dsig} definiert eine Menge von
Schlüsseltypen, unter anderem für \gls{x509} und \gls{openpgp}, die jedoch auch abhängig von der Anwendung mit Hilfe neuer \gls{xml}-Namespaces erweitert werden
können. Das in \fref{lst:xml-dsig-keyinfo} dargestellte \gls{xml}-Schema zeigt die bereits definierten Elemente von \gls{xml-dsig} zur Aufnahme von
Schlüsselinformationen.

\lstinputlisting[language=XML,caption={Schema von \protect\xmlelem{KeyInfo}},label=lst:xml-dsig-keyinfo,emph={element}]{source/xml-dsig-keyinfo.xml}

Ist \xmlelem{KeyInfo} nicht Bestandteil der Signatur, dann wird davon ausgegangen, dass der Empfänger über den Kontext der Nachricht und der Signatur in der
Lage ist, die nötigen Schlüsselinformationen zu beziehen.

Über \xmlelem{KeyName} ist es dem Ersteller der Signatur möglich, den Schlüssel zu benennen, der für die Signatur benutzt wurde. Für \gls{x509} kann dies ein
\gls{dn} sein, für \gls{openpgp} eine Email-Adresse. Über \xmlelem{KeyValue} ist es auch möglich, die für den \gls{dsa} oder das \gls{rsa} Verfahren
benötigten, Schlüsselinformationen direkt in der Signatur einzubetten.

\subsection{OpenPGP}
Für Schlüssel nach \gls{openpgp} wird das Element \xmlelem{PGPData} im \gls{xml-ns} \url{http://www.w3.org/2000/09/xmldsig#PGPData} definiert. Es ist in der
Lage, zwei weitere Elemente aufzunehmen: \xmlelem{PGPKeyID} nimmt die Schlüssel-ID eines \gls{openpgp}-Schlüsselpaares auf, welche aus 8 Okteten besteht und die
niedrigsten 64 Bit aus der \gls{sha1} Prüfsumme des Schlüssels darstellt \cite{rfc2440:ietf}. Über die \xmlelem{PGPKeyID} ist der passende Schlüssel bereits eindeutig
identifiziert, es besteht mit dem Element \xmlelem{PGKeyPacket} jedoch auch die Variante, den vollständigen öffentlichen Teil des Schlüsselpaares in der
Signatur einzubetten. Dafür wird das Schlüsselmaterial mit \gls{base64} kodiert und als Inhalt in das Element geschrieben \cite{xml-dsig:w3c}.
\fref{lst:xml-dsig-keyinfo-openpgp} zeigt ein Beispiel für eine Struktur mit den zuvor genannten Unterelementen, welche auf den \gls{openpgp}-Schlüssel des Autors
referenzieren.

\lstinputlisting[language=XML,caption={Beispiel einer Struktur für \protect\xmlelem{PGPData}},label=lst:xml-dsig-keyinfo-openpgp,emph={element}]{source/xml-dsig-keyinfo-openpgp.xml}

\subsection{X.509}
Informationen über ein für die Signatur verwendetes \gls{x509}-Zertifikat können unterhalb des Elements \xmlelem{X509Data} im \gls{xml-ns}
\url{http://www.w3.org/2000/09/xmldsig#X509Data} abgelegt werden. 
Ähnlich wie \xmlelem{PGPData} enthält es Schlüsselinformationen die zur Validierung der Signatur benötigt werden.  Dafür muss mindestens eines der Elemente aus
\fref{tab:x509data-elements} vorhanden sein. Wenn zwei oder mehrere dieser Elemente sich innerhalb eines \xmlelem{X509Data} befinden, so müssen sie alle das
selbe Zertifikat referenzieren. In \fref{lst:xml-dsig-keyinfo-x509} sind drei Zertifikate über unterschiedliche Methoden referenziert. 

\begin{table}
    \centering
    \begin{tabularx}{\textwidth}{ l X }
        Element  & Inhalt \\
        \hline
        \hline
        \xmlelem{X509IssuerSerial} & Name und Seriennummer des Zertifikats \\
        \hline
        \xmlelem{X509SKI} & Der base64 kodierte Inhalt der "`SubjectKeyIdentifier"' Erweiterung des Zertifikats \\
        \hline
        \xmlelem{X509Certificate} & Der base64 kodierte Inhalt des gesamten Zertifikats \\
        \hline
        \xmlelem{X509SubjectName} & \gls{dn} des Subject-Felds des Zertifikats \\
        \hline
    \end{tabularx}
    \caption{Elemente in \xmlelem{X509Data} und ihr Inhalt}
    \label{tab:x509data-elements}
\end{table}

\lstinputlisting[float=bt,language=XML,caption={Beispiel einer Struktur für \protect\xmlelem{X509Data}},label=lst:xml-dsig-keyinfo-x509,emph={X509Data}]{source/xml-dsig-keyinfo-x509.xml}

\section{Signatureinformation}

\section{Referenzen}
\index{Referenzen}
\label{sec:Signaturbindung:Referenzen}

\section{Transformationen}
\index{Transformationen}
\label{sec:Signaturbindung:Transformationen}

\subsection{XPath}
\index{XPath}
\label{sec:Signaturbindung:Transformationen:XPath}

\subsection{XSLT}
\index{XSLT}
\label{sec:Signaturbindung:Transformationen:XSLT}


\section{Bindung an ein XHTML-Dokument}
Eine Signatur mit \gls{xml-dsig} hält ihre Verweise zu den signierten Inhalten über die Reference-Elemente. Im Fall der Verwendung innerhalb eines
\gls{webbrowser} muss das aufgerufene \gls{xhtml}-Dokument einen Verweis auf die Signatur beinhalten. Dies ist notwendig, da die Information über die Bereiche,
welche signiert wurden, nur in der Signatur selbst festgehalten werden. Ruft ein \gls{webbrowser} ein \gls{xhtml}-Dokument ab, so muss ihm bekannt gemacht
werden, dass für dieses Dokument Signaturinformationen vorhanden sind. Sind diese Informationen verfügbar gemacht worden, kann der \gls{webbrowser} daraus die
signierten Bereiche ermitteln.

Um solche Verweise auf Signaturen an die \gls{xhtml} binden zu können, bieten sich mehrere Verfahren an.

\subsection{HTTP-Header}
Eigener X-HTTP-Header wird vom Webserver ausgeliefert, worin über \glspl{url} auf Signatur-Dateien verwiesen wird. Keine Auswirkung auf XHTML, jedoch praktische Beschränkung
auf Anzahl der \glspl{url}?

\subsection{\xmlelem{link}-Element}
Der \gls{xhtml}-Standard definiert das \xmlelem{link}-Element, um Relationen zu anderen Dokumenten zu definieren \cite{xhtml:oreilly}. Es ist Teil des
\xmlelem{head}-Bereichs und kann dort mehrfach eingesetzt werden. Ein einzelnes \xmlelem{link}-Element definiert genau eine Relation. Bezüglich des Verhaltens
der \gls{webbrowser} beim Verarbeiten eines \xmlelem{link}-Elements macht der \gls{xhtml}-Standard keine verbindlichen Aussagen. Es wird aber darauf
hingewiesen, dass dieses Element speziell für zukünftige, noch nicht definierte Verhaltensmuster bei \gls{webbrowser} vorgesehen ist.

Es ist damit möglich, über das \xmlattr{href}-Attribut des \xmlelem{link}-Elements den Verweis auf die \gls{url} einer Signaturinformation zu realisieren. In
\fref{lst:link} ist dies auf \fref{lin:link-href} zu sehen. Da das
Element auch mehrfach in einem \gls{xhtml}-Dokument verwendet werden kann, lassen sich damit auch mehrere unterschiedliche Signaturinformationen einbinden.

\lstinputlisting[language=HTML,caption={Verwendung von \xmlelem{link}},label=lst:link,emph={link,href}]{source/link.html}

\subsection{XML-Namespaces}
Direktes Einbetten der Signatur in das Dokument selbst.

\lstinputlisting[language=HTML,caption={Verwendung von XML-Namespaces},label=lst:xmlns,emph={Signature}]{source/xmlns.html}

\subsection{XInclude}
XInclude ist eine neue Technologie, die vom W3C entwickelt wurde, um mehrere \gls{xml}-Dokumente in ein gemeinsames Dokument zusammenzuführen \cite{xml:oreilly}.
Dabei wird das \xmlelem{include}-Element aus dem reservieren XML-Namespace für \gls{xinclude} benutzt um über \gls{uri} externe \gls{xml}-Dokumente während der
Auswertung des Dokuments einzubinden.

\lstinputlisting[language=HTML,caption={Verwendung von XInclude},label=lst:xinclude,emph={include,xi}]{source/xinclude.html}
