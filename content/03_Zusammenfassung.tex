% vim: set tw=160:

\chapter{Zusammenfassung}
\label{chap:Zusammenfassung}
\gls{xml-dsig} kann aufgrund seinen flexiblen Art, Informationen zu signieren, als adequate Lösung für das Problem der Einbettung von Signaturen in \gls{xhtml}
betrachtet werden. Mit der Kombination aus Transformationen und Kanonisierung der \gls{xml}-Strukturen können Informationen effizent zusammengeführt werden um
sie zu signieren.

Um die Präsenz einer Signatur in einem \gls{xhtml}-Dokument anzuzeigen eigenen sich Mikroformate und das \xmlelem{link}-Element besonders, da sie die geringsten 
Anforderungen an die verwendete \gls{xml}-Schnittstelle stellen. Besonders Mikroformate können durch die Vorgabe von Transformationen und auch Referenzen mit 
nur geringen Auswirkungen in Bezug auf die Größe in ein \gls{xhtml}-Dokument eingebettet werden.

Da weder \gls{openpgp} noch \gls{x509} in beiden Anwendungsbereichen, privat wie unternehmerisch, einen deutlichen Vorteil gegenüber dem jeweils anderen
Verfahren vorweisen können, wäre es wünschenswert, in einem \gls{firefox-addon} beide Verfahren zu unterstützen. Dies kann durch die abstrakte
Programmierschnittstelle im AddOn erreicht werden, über die je nach Signatur entweder \gls{openpgp} oder \gls{x509} zur Prüfung eingesetzt werden können.

