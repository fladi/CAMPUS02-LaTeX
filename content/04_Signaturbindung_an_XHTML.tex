\chapter{Bindung an ein XHTML-Dokument}
\label{sec:Signaturbindung:XHTML}
Eine Signatur mit \gls{xml-dsig} hält ihre Verweise zu den signierten Inhalten über die \xmlelem{Reference}-Elemente aus \fref{sec:Signaturbindung:Referenzen}.
Im Fall der Verwendung innerhalb eines \gls{webbrowser} muss das aufgerufene \gls{xhtml}-Dokument einen Verweis auf die Signatur beinhalten. Dies ist notwendig,
da die Information über die Bereiche, welche signiert wurden, nur in der Signatur selbst festgehalten werden. Ruft ein \gls{webbrowser} ein \gls{xhtml}-Dokument
ab, so muss ihm bekannt gemacht werden, dass für dieses Dokument Signaturinformationen vorhanden sind. Sind diese Informationen verfügbar gemacht worden, kann
der \gls{webbrowser} daraus die signierten Bereiche ermitteln.

Um solche Verweise auf Signaturen an die \gls{xhtml} binden zu können, bieten sich mehrere Verfahren an.

\section{HTTP-Header}
\index{HTTP-Header}
Als Antwort auf die anfrage eines \gls{webbrowser} wird vom Webserver ein eigner \gls{http}-Header ausgeliefert, welcher eine Liste von \glspl{url} enthält, die auf
Signatur-Dateien verweisen. Dadurch wäre eine klare Trennung der Signaturen vom \gls{xhtml}-Inhalt gewährleistet.
Jedoch ist noch unklar ob eine praktische Beschränkung auf die Größe eines HTTP-Headers besteht.\todo

\section{\xmlelem{link}-Element}
Der \gls{xhtml}-Standard definiert das \xmlelem{link}-Element, um Relationen zu anderen Dokumenten zu definieren \cite{xhtml:oreilly}. Es ist Teil des
\xmlelem{head}-Bereichs und kann dort mehrfach eingesetzt werden. Ein einzelnes \xmlelem{link}-Element definiert genau eine Relation. Bezüglich des Verhaltens
der \gls{webbrowser} beim Verarbeiten eines \xmlelem{link}-Elements macht der \gls{xhtml}-Standard keine verbindlichen Aussagen. Es wird aber darauf
hingewiesen, dass dieses Element speziell für zukünftige, noch nicht definierte Verhaltensmuster bei \gls{webbrowser} vorgesehen ist.

Es ist damit möglich, über das \xmlattr{link}{href}-Attribut des \xmlelem{link}-Elements den Verweis auf die \gls{url} einer Signaturinformation zu realisieren. In
\fref{lst:link} ist dies auf \fref{lin:link-href} zu sehen. Da das
Element auch mehrfach in einem \gls{xhtml}-Dokument verwendet werden kann, lassen sich damit auch mehrere unterschiedliche Signaturinformationen einbinden.

\lstinputlisting[language=HTML,caption={Verwendung von \xmlelem{link}},label=lst:link,emph={link,href}]{source/link.html}

\section{Microformat}
\todo

\section{XML-Namespaces}
Direktes Einbetten der Signatur in das Dokument selbst.

\lstinputlisting[language=HTML,caption={Verwendung von XML-Namespaces},label=lst:xmlns,emph={Signature}]{source/xmlns.html}

\section{XInclude}
XInclude ist eine neue Technologie, die vom W3C entwickelt wurde, um mehrere \gls{xml}-Dokumente in ein gemeinsames Dokument zusammenzuführen \cite{xml:oreilly}.
Dabei wird das \xmlelem{include}-Element aus dem reservieren \gls{xml-ns} für \gls{xinclude} benutzt um über das \xmlattr{xinclude}{href}-Attribut den \gls{uri}
eines externen \gls{xml}-Dokuments während der Auswertung des Haupt-Dokuments einzubinden.

\lstinputlisting[language=HTML,caption={Verwendung von XInclude},label=lst:xinclude,emph={include,xi}]{source/xinclude.html}
